%%%%%%%%%%%%%%%%%%%%%%%%%%%%%%%%%%%%%%%%%%%%%%%%%%%%%%%%%%%%%%%%%%%%%%
% LaTeX Example: Project Report
%
% Source: http://www.howtotex.com
%
% Feel free to distribute this example, but please keep the referral
% to howtotex.com
% Date: March 2011 
% 
%%%%%%%%%%%%%%%%%%%%%%%%%%%%%%%%%%%%%%%%%%%%%%%%%%%%%%%%%%%%%%%%%%%%%%
% How to use writeLaTeX: 
%
% You edit the source code here on the left, and the preview on the
% right shows you the result within a few seconds.
%
% Bookmark this page and share the URL with your co-authors. They can
% edit at the same time!
%
% You can upload figures, bibliographies, custom classes and
% styles using the files menu.
%
% If you're new to LaTeX, the wikibook is a great place to start:
% http://en.wikibooks.org/wiki/LaTeX
%
%%%%%%%%%%%%%%%%%%%%%%%%%%%%%%%%%%%%%%%%%%%%%%%%%%%%%%%%%%%%%%%%%%%%%%
% Edit the title below to update the display in My Documents
%\title{Project Report}
%
%%% Preamble
\documentclass[paper=a4, fontsize=11pt]{scrartcl}
\usepackage[T1]{fontenc}
\usepackage{fourier}

\usepackage[english]{babel}                                                         % English language/hyphenation
\usepackage[protrusion=true,expansion=true]{microtype}  
\usepackage{amsmath,amsfonts,amsthm} % Math packages
\usepackage[pdftex]{graphicx}   
\usepackage{url}


%%% Custom sectioning
\usepackage{sectsty}
\allsectionsfont{\centering \normalfont\scshape}


%%% Custom headers/footers (fancyhdr package)
\usepackage{fancyhdr}
\pagestyle{fancyplain}
\fancyhead{}                                            % No page header
\fancyfoot[L]{}                                         % Empty 
\fancyfoot[C]{}                                         % Empty
\fancyfoot[R]{\thepage}                                 % Pagenumbering
\renewcommand{\headrulewidth}{0pt}          % Remove header underlines
\renewcommand{\footrulewidth}{0pt}              % Remove footer underlines
\setlength{\headheight}{13.6pt}


%%% Equation and float numbering
\numberwithin{equation}{section}        % Equationnumbering: section.eq#
\numberwithin{figure}{section}          % Figurenumbering: section.fig#
\numberwithin{table}{section}               % Tablenumbering: section.tab#


%%% Maketitle metadata
\newcommand{\horrule}[1]{\rule{\linewidth}{#1}}     % Horizontal rule

\title{
        %\vspace{-1in}  
        \usefont{OT1}{bch}{b}{n}
        \normalfont \normalsize \textsc{MIT 18.086 Computational Science and Engineering II} \\ [25pt]
        \horrule{0.5pt} \\[0.4cm]
        \huge Computational Fluid Dynamics with Shocks \\
        \horrule{2pt} \\[0.5cm]
}
\author{
        \normalfont                                 \normalsize
        Matthew Vernacchia\\[-3pt]      \normalsize
        \today
}
\date{}


%%% Begin document
\begin{document}
\maketitle
\section{Abstract}

%%%%%%%%%%%%%%%%%%%%%%%%%%%%%%%%%%%%%%%%%%%%%%%%%%%%%%%%%%%%%%%%%%%%%%%%%%%%%%%
\section{Introduction}
\subsection{Motivation}
Many engineering problems in the military and transportation sectors involve supersonic fluid flow. A significant phenomenon in many supersonic flow problems is the formation of shock waves. Important design problems, such as a helmets to protect against traumatic brain injury from explosions, rocket nozzles for missiles and launch vehicles, or a spacecraft's reentry trajectory, depend on our ability to model and understand shocks.\\
Analytical solutions exist for some simple problems, and these were the primary engineering analysis tools during the 1940s, 50s and 60s. However, these tools restricted designers to simple geometries and a piecewise understanding of the problem. Consider an aircraft designer asked to predict the shock pattern and drag of a fighter jet in supersonic flight. He could solve the shock pattern from a conical nose and from a diamond-section wing or tail, and add these results together to get a rough estimate of the total vehicle's drag. However, the interaction of these shocks where the wing and fuselage meet would be left to guesswork. And a more complicated wing cross section (i.e. a proper airfoil) would not be possible to solve. Therefore, a drag prediction of useful fidelity would require expensive and time consuming wind tunnel tests.\\
Numerical solutions to fluid flow problems (Computational Fluid Dynamics) now enable engineers to analyze more complicated problems. This allows for the design of higher-performance vehicles. Although wind tunnel testing is often still necessary to calibrate or validate CFD results, the use of CFD means that much fewer physical experiments must be performed. This can substantially reduce the cost and time of development.  


\subsection{Governing Equations}

\subsection{Numerical Techniques}

%%%%%%%%%%%%%%%%%%%%%%%%%%%%%%%%%%%%%%%%%%%%%%%%%%%%%%%%%%%%%%%%%%%%%%%%%%%%%%%
\section{Implementation}
\subsection{Computation Platform}

\subsection{Solver}
\subsubsection{Time Update: MacCormack Scheme}

\subsubsection{Shock Capturing: Artificial Dissipation}

\subsubsection{Curvilinear Coordinates}

\subsubsection{Boundary Conditions}


%%%%%%%%%%%%%%%%%%%%%%%%%%%%%%%%%%%%%%%%%%%%%%%%%%%%%%%%%%%%%%%%%%%%%%%%%%%%%%%
\section{Results}
\subsection{Shock Tube}

\subsection{Mach $1.5$ Ramp}

%%%%%%%%%%%%%%%%%%%%%%%%%%%%%%%%%%%%%%%%%%%%%%%%%%%%%%%%%%%%%%%%%%%%%%%%%%%%%%%
\section{Conclusion}

\end{document}